% definit le type de document et ses options
\documentclass[a4paper,11pt]{article}


% des paquetages indispensables, qui ajoutent des fonctionnalites
\usepackage[utf8]{inputenc}
\usepackage[T1]{fontenc}
\usepackage{amsmath,amssymb,amsfonts}
\usepackage{fullpage}
\usepackage{graphicx}
\usepackage{url}
\usepackage{xspace}
\usepackage[francais]{babel}
\usepackage{setspace}
\usepackage{color}
\usepackage{multirow}
\usepackage[normalem]{ulem}
\usepackage[final]{pdfpages}
\usepackage{stmaryrd}
\usepackage{algorithm}
\usepackage{algorithmic}


\usepackage[left=1.5cm,right=1.5cm,top=2cm,bottom=2cm]{geometry}

%%Couleurs des licornes
\definecolor{darkblue}{RGB}{15, 5, 102}
\definecolor{darkgreen}{RGB}{0, 86, 27}
\definecolor{darkred}{RGB}{150,0,24}


% pour ecrire les reponses
\newcounter{nreponse}
\newenvironment{reponse}
{\refstepcounter{nreponse} \begin{normalsize} \normalfont \textbf{\textcolor{darkgreen}{Question \arabic{nreponse}   :}} \newline } {\normalsize \end{normalsize}}
%Sous-réponses
\newcounter{nsubreponse}
\newenvironment{subreponse}
{\refstepcounter{nsubreponse} \begin{normalsize} \normalfont \textbf{ \textcolor{darkgreen}{(\alph{nsubreponse})  :}} \newline } {\normalsize \end{normalsize}}


%Partie
\addto\captionsfrench{
	\renewcommand{\partname}{Partie}
	\renewcommand{\thepart}{\Roman{part}}
}
%Barre verticale
\newcommand\horizon{\setlength\unitlength{\textwidth}
	\line(1,0){1} 
	\newline}


% des commandes utiles pour ecrire des maths : rajoutez les votres!
\newcommand{\dx}{\,dx}
\newcommand{\ito}{,\dotsc,}
\newcommand{\R}{\mathbb{R}}
\newcommand{\C}{\mathbb{C}}
\newcommand{\N}{\mathbb{N}}
\newcommand{\Poly}[1]{\mathcal{P}_{#1}}
\newcommand{\abs}[1]{\left\lvert#1\right\rvert}
\newcommand{\norm}[1]{\left\lVert#1\right\rVert}
\newcommand{\pars}[1]{\left(#1\right)}
\newcommand{\bigpars}[1]{\bigl(#1\bigr)}
\newcommand{\set}[1]{\left\{#1\right\}}


%petit et grand tau
% Source for different sizes :
%     http://forum.mathematex.net/latex-f6/bonnes-commandes-de-base-t12278.html
\newcommand{\bigO}[1]{\ensuremath{\mathop{}\mathopen{}\mathcal{O}\mathopen{}\left(#1\right)}}

\newcommand\smallO[1]{
	\mathchoice
	{% mode \displaystyle
		\ensuremath{\mathop{}\mathopen{}{\scriptstyle\mathcal{O}}\mathopen{}\left(#1\right)}
	}
	{% mode \textstyle
		\ensuremath{\mathop{}\mathopen{}{\scriptstyle\mathcal{O}}\mathopen{}\left(#1\right)}
	}
	{% mode \scriptstyle
		\ensuremath{\mathop{}\mathopen{}{\scriptscriptstyle\mathcal{O}}\mathopen{}\left(#1\right)}
	}
	{% mode \scriptscriptstyle
		\ensuremath{\mathop{}\mathopen{}{o}\mathopen{}\left(#1\right)}
	}
}
%Style mathématiques pour grandes intégrales
\newcommand{\ds}{\displaystyle}
\everymath{\displaystyle}


% titre, auteur et date
% titre, auteur et date
\title{\huge \textbf{{\textcolor{darkblue}{Optimisation numérique\\ \LARGE Comparaison de méthodes de calcul de minimum}}}}
\author{\large \textbf{{\textcolor{darkblue}{Guillaume DELORME, Mickaël LY}}}}
\date{\Large \textbf{\textcolor{darkblue}{\today}}}


%%bas de page
\usepackage{fancyhdr}
\pagestyle{fancy}

\renewcommand{\headrulewidth}{0pt}

\fancyfoot[L]{\textbf{{Charte d'équipe - page \thepage}}}
\fancyfoot[C]{}
\fancyfoot[R]{\textbf{{Hammen, Kacher, Ly, Stoffel}}}
\renewcommand{\footrulewidth}{1pt}

\newcommand{\puce}{\renewcommand{\labelitemi}{\textbullet}}


% le debut du contenu
%===============
\begin{document}
	\begin{spacing}{1.3}%agrandir l'interline
		%===============
		\thispagestyle{fancy}
		% pour afficher titre, auteur et date

		\begin{minipage}{0.33\linewidth}
			\centering
			\Large Maxence HAMMEN \\
			\normalsize 2A Ensimag MMIS \\
			\small maxence.hammen@ensimag.grenoble-inp.fr\\
			\small 06 16 82 94 22
		\end{minipage}
		\begin{minipage}{0.33\linewidth}
		  %% Empty
                  ~~~
		\end{minipage}
		\begin{minipage}{0.33\linewidth}
			\centering
			\Large Ilyes KACHER \\
			\normalsize 2A Ensimag ISI \\
			\small ilyes.kacher@ensimag.grenoble-inp.fr\\
			\small 06 63 39 72 21
		\end{minipage}\\[0.5cm]

		
		\begin{minipage}{0.33\linewidth}
			\centering
			\Large Mickaël LY\\
			\normalsize 2A Ensimag MMIS\\
			\small mickael.ly@ensimag.grenoble-inp.fr\\
			\small 06 37 18 42 61
		\end{minipage}
                \begin{minipage}{0.33\linewidth}
		  %% Empty
                  ~~~
		\end{minipage}
		\begin{minipage}{0.33\linewidth}
			\centering
			\Large Mathieu STOFFEL \\
			\normalsize 2A Ensimag ISI \\
			\small mathieu.stoffel@ensimag.grenoble-inp.fr\\
			\small 06 98 89 84 99
		\end{minipage}\\

		\begin{center}
			
			\Huge \underline{Charte d'équipe}
			
		\end{center} ~~\\

		\vspace{-1cm}
		
		\section*{Préambule}

		Cette charte que nous avons tous quatre élaborée a pour buts à la fois d'organiser et de garantir la bonne organisation du groupe durant le projet de spécialité, autant en définissant des règles encadrant le travail de chacun que des règles destinées à conserver et entretenir la bonne humeur au sein du groupe.\\
	
		En souscrivant donc à la présente, chaque membre s'engage donc à fournir travail sérieux tout en préservant l'entente générale afin que chacun puisse s'épanouir durant ce projet.~~\\
		
		
		\section*{Contenu de la charte}
		
		\subsection*{Organisation générale}

		\begin{tabular}{p{0.2\linewidth}  p{0.005\linewidth} p{0.7\linewidth} }
			$\bullet$ Jours de travail & & La présence de chaque membre du groupe est demandée chaque jour du lundi au vendredi. Sauf décision prise à la majorité ou sur la base du volontariat, les week-ends sont réservés au repos et éventuellement à la réflexion sur le travail fait et à faire. \\[0.3cm]
			$\bullet$ Horaires & & Durant les jours de travail, chaque membre est tenu d'être présent 8 heures, en particuliers durant les 5 heures de présences communes correspondant aux créneaux 9h30-11h30 et 14h-17h.\\[0.3cm]
			$\bullet$ Nuits blanches & & Dans le but de rentabiliser les heures de travail communes, chaque membre se doit d'être en forme. C'est pourquoi nous proscrivons le travail de nuit, sauf cas exceptionnels décidés ensemble. \\[0.3cm]
			$\bullet$ Pause-déjeuner & & Dans le même objectif que la règle précédente, la pause déjeuner est obligatoire. \\[0.3cm]
			
		\end{tabular}

		\subsection*{Organisation générale (suite)}

		\begin{tabular}{p{0.2\linewidth} p{0.005\linewidth} p{0.7\linewidth} }
		$\bullet$ Petit-déjeuner & & Suivant le principe du 'gâteauscope', lui-même dérivé du 'colloscope' de la prépa, calendrier organisant les 'colles', un 'petitdejoscope' sera mis en place au sein de l'équipe. Tout manquement à ce petitdejoscope sera puni d'une majoration du nombre de viennoiseries à amener, auquel peut s'ajouter une bouteille de Champomy à la fin du projet.\\[0.3cm]
		$\bullet$ Courtoisie & & Toute règle de courtoisie élémentaire est évidemment de mise dans l'équipe.\\[0.3cm]
		$\bullet$ Retards et absences & & Tout manquement aux horaires de travail doit être signalé le plus tôt possible aux autres membres du groupe et devra être justifié.\\[0.3cm]
		\end{tabular}
		

                \subsection*{Organisation de l'équipe - Gestion de l'agilité}

                \begin{tabular}{p{0.2\linewidth}    p{0.005\linewidth}
                    p{0.7\linewidth} }
		  $\bullet$   Roles  dans   l'équipe   &   &  Suite à une concertation, les responsabilités suivantes ont été attribuées : \newline
                  \indent \indent \begin{tabular}{p{0.1\linewidth} l l}
                    &Mathieu & Responsable Git\\
                    &Mickael & Responsable documentation, secrétaire\\
                    &Ilyes   & Scrum Master\\
                    &Maxence & Responsable tests\\
                  \end{tabular}
                  \\[0.3cm]
                  $\bullet$ Stand-up  & & Chaque matin  est prévue une
                  réunion  Stand-up à  l'aide d'une  table des  taches
                  virtuelles via l'utilitaire \emph{Trello}.\\[0.3cm]
		  $\bullet$  Sprints &  & Le  fonctionnement théorique
                  prévu  consiste en  des sprints  d'une durée modulable selon  les taches
                  prévues. Ces sprints seront séparés par des réunions
                  afin de définir les objectifs
                  du sprint.\\[0.3cm]
		  $\bullet$  Peer-coding &  &  Sauf points  techniques
                  délicats,  nous préfererons  coder chacun  sur notre
                  machine.\\
                  

                \end{tabular}
		

                  
		\subsection*{Règles de travail - Code}
		
		\begin{tabular}{p{0.2\linewidth} p{0.005\linewidth} p{0.7\linewidth} }
			
		  $\bullet$ Critiques & & Chaque membre s'engage à formuler des critiques uniquement constructives envers ses co-développeurs, et s'engage aussi à intégrer ce type de critiques afin de s'améliorer. Toute autre forme de critique est proscrite.\\[0.3cm]
		  $\bullet$ Communication & & Chaque membre est fortement invité à communiquer ses idées, ses désaccords, et ce avec tous les membres de l'équipe via le mode de communication qu'il jugera le plus approprié (oral, mail, téléphone...)\\[0.3cm]
		  $\bullet$ Commentaires & & Afin d'assurer la lisibilité et la compréhension du code par tous, chaque membre du groupe s'engage à commenter continuellement son code.\\
		
		
		\end{tabular}



                  
		\subsection*{Règles de travail - Git}
		
		\begin{tabular}{p{0.2\linewidth} p{0.005\linewidth} p{0.7\linewidth} }
		
                  $\bullet$ Branches de travail & & Chaque membre sera restreint à une branche de développement du projet selon la répartition des tâches qui sera décidée.\\[0.3cm]
		  $\bullet$ Branche Master & & Cette branche représente le travail commun de l'ensemble du groupe. Toute modification majeure de cette branche doit donc être soumise à l'approbation du groupe.\\[0.3cm]
		  $\bullet$ Push & & Chaque membre s'engage à ne pas push de code ne compilant pas sans autorisation du groupe (relecture commune par exemple).\\[0.3cm]
		$\bullet$  Commit   &  &  Les  commits   devront  etre
                  réguliers et  de taille raisonnable (\~ 50  lignes de
                  code). Ils devront aussi  respecter le format imposé
                  : 80  caractères au  plus de  résumé du  commit, une
                  ligne  vide et  enfin des  paragraphes d'explication
                  plus détaillés.

                \end{tabular}
		
		%Pour ne pas numéroter la page de garde
		%\setcounter{page}{0}
		\thispagestyle{fancy}
		%\newpage

		
	\end{spacing}
\end{document}
