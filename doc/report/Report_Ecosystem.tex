% definit le type de document et ses options
\documentclass[a4paper,12pt,twoside]{article}

% des paquetages indispensables, qui ajoutent des fonctionnalites
\usepackage[utf8]{inputenc}
\usepackage[T1]{fontenc}
\usepackage{amsmath,amssymb,amsfonts}
\usepackage{fullpage}
\usepackage{graphicx}
\usepackage{url}
\usepackage{xspace}
\usepackage[english]{babel}
\usepackage{setspace}
\usepackage{color}
\usepackage{multirow}
\usepackage[normalem]{ulem}
\usepackage[final]{pdfpages}
\usepackage{stmaryrd}
\usepackage{caption}


\usepackage[left=2.5cm,right=2.5cm,top=2.5cm,bottom=2.5cm]{geometry}

%%Couleurs des licornes
\definecolor{darkblue}{RGB}{8, 3, 51}
\definecolor{darkgreen}{RGB}{0, 33, 11}
\definecolor{darkred}{RGB}{150,0,24}


%Partie
\addto\captionsfrench{
	\renewcommand{\partname}{Partie}
	\renewcommand{\thepart}{\Roman{part}}
}
%Barre verticale
\newcommand\horizon{\setlength\unitlength{\textwidth}
	\line(1,0){1} 
	\newline}


% des commandes utiles pour ecrire des maths : rajoutez les votres!
\newcommand{\dx}{\,dx}
\newcommand{\ito}{,\dotsc,}
\newcommand{\R}{\mathbb{R}}
\newcommand{\C}{\mathbb{C}}
\newcommand{\N}{\mathbb{N}}
\newcommand{\Poly}[1]{\mathcal{P}_{#1}}
\newcommand{\abs}[1]{\left\lvert#1\right\rvert}
\newcommand{\norm}[1]{\left\lVert#1\right\rVert}
\newcommand{\pars}[1]{\left(#1\right)}
\newcommand{\bigpars}[1]{\bigl(#1\bigr)}
\newcommand{\set}[1]{\left\{#1\right\}}



% titre, auteur et date
\title{\textbf{{\textcolor{darkblue}{{\Huge Project of speciality - Ensimag 2015/2016}\\[2cm] {\huge Generation, simulation and animation of an ecosystem}}}}}
\author{\large \textbf{{\textcolor{darkblue}{}}}}
\date{\Large \textbf{\textcolor{darkblue}{}}}


%%bas de page
\usepackage{fancyhdr}
\pagestyle{fancy}
\usepackage{emptypage}
\renewcommand{\headrulewidth}{0pt}

\fancyhead{}
\renewcommand{\footrulewidth}{0pt}


\fancyfoot[C]{}
\fancyfoot[RO]{\bfseries\textcolor{darkblue}{\thepage}}
\fancyfoot[LO]{\bfseries\textcolor{darkblue}{Hammen - Kacher - Ly - Stoffel}}

\fancyfoot[LE]{\bfseries\textcolor{darkblue}{\thepage}}
\fancyfoot[RE]{\bfseries\textcolor{darkblue}{Hammen - Kacher - Ly - Stoffel}}
	

\newcommand{\puce}{\renewcommand{\labelitemi}{\textbullet}}


\setcounter{secnumdepth}{3}
\setcounter{tocdepth}{3}

\usepackage{xcolor}
\usepackage{sectsty}
\partfont{\Huge\color{darkgreen}}  % sets colour of chapters
\chapterfont{\color{darkgreen}\thispagestyle{fancy}} 
\sectionfont{\color{darkgreen}} 
\subsectionfont{\color{darkgreen}}
\subsubsectionfont{\color{darkgreen}}

\usepackage{titlesec}

%\titleformat*{\chapter}{\color{darkgreen}\bfseries\Large}
%\titleformat*{\subsubsection}{\color{darkgreen}\bfseries\Large}

\titlespacing*{\chapter}
{0pt}{5.5ex plus 1ex minus .2ex}{4.3ex plus .2ex}
\titlespacing*{\section}
{0pt}{5.5ex plus 1ex minus .2ex}{4.3ex plus .2ex}
\titlespacing*{\subsection}
{0pt}{5.5ex plus 1ex minus .2ex}{4.3ex plus .2ex}
\titlespacing*{\subsubsection}
{0pt}{5.5ex plus 1ex minus .2ex}{4.3ex plus .2ex}

\usepackage{indentfirst}



\usepackage{changepage,ifthen}
\newcommand\skiptoevenpage{%
	\checkoddpage
	\ifthenelse{\boolean{oddpage}}%
	{\null\clearpage}%
	{\null\clearpage \null \clearpage}%
}

\newcommand\skiptooddpage{%
	\checkoddpage
	\ifthenelse{\boolean{oddpage}}%
	{\null\clearpage \null \clearpage}%
	{\null\clearpage}%
}
% le debut du contenu
%===============
\begin{document}
	\begin{spacing}{1.3}%agrandir l'interline
	\setcounter{page}{1}
	\begin{center}
	% Front page
	~~\\
	\vspace{0cm}
	
	\begin{minipage}{\linewidth}
		\maketitle
	\end{minipage}	
~~\\
	
	\begin{flushright}

	\begin{minipage}{0.5\linewidth}
		%\includegraphics[height=6cm]{chess.png}	
		~~
		\vspace{5cm}	
		Here a badass picture of our project
	\end{minipage}
	\end{flushright}
	
	\vspace{2cm}
	
	\begin{flushright}
	
	\begin{minipage}{0.5\linewidth}
	\begin{tabular}{|l l l}
		\hspace{0.1cm}&\huge \textcolor{darkblue}{Hammen} & \huge \textcolor{darkblue}{Maxence} \\
		\hspace{0.1cm}&\huge \textcolor{darkblue}{Kacher} & \huge \textcolor{darkblue}{Ilyes} \\
		\hspace{0.1cm}&\huge \textcolor{darkblue}{Ly} & \huge \textcolor{darkblue}{Mickaël} \\
		\hspace{0.1cm}&\huge \textcolor{darkblue}{Stoffel} & \huge \textcolor{darkblue}{MaThieu} \\
	\end{tabular}
	\end{minipage}
	\end{flushright}
	\end{center}
		
	\thispagestyle{empty}
	% % % % % % % % % % % % % % % % %
	\newpage
	% % Blank page
	~~\\
	\thispagestyle{empty}
	% % % % % % % % % % % % % % %
	\newpage
	~~\\
	%\pagestyle{empty}
	\tableofcontents
	\pagestyle{empty}
	\newpage


	% % % % % % % % % % % % % % % % % % % % % % % % % % % % % % % % % % % %
	% % % % % % % % % % % % % % % % % % % % % % % % % % % % % % % % % % % %
	% % %
	% % %  Beginning of the content
	% % %
	% % % % % % % % % % % % % % % % % % % % % % % % % % % % % % % % % % % %
	% % % % % % % % % % % % % % % % % % % % % % % % % % % % % % % % % % % %


	\skiptooddpage
	\setcounter{page}{1}
	\pagestyle{fancy}
	
	
	~~
	\vspace{1cm}
	
	\part{Introduction}
	~~
	\vspace{1cm}
	
	Unleash your power here to deploy the most beautiful man-made introduction.\\ 
	
	Briefly explain here the context of the project of speciality, the goal and the content of the project -ie what've sent to the teachers.
	Then introduce the following parts : environment and boids

	% % % % % % % % % % % % % % % % % % % % % % % % % % % % % % % % % % % %
	% % % % % % % % % % % % % % % % % % % % % % % % % % % % % % % % % % % %
	% % %
	% % %  Environment part
	% % %
	% % % % % % % % % % % % % % % % % % % % % % % % % % % % % % % % % % % %
	% % % % % % % % % % % % % % % % % % % % % % % % % % % % % % % % % % % %


	\newpage
	\skiptooddpage
	~~
	\vspace{1cm}

	\part{Core part i don't know the title yet}
	~~
	
	\vspace{1cm}
	\section{A quick introduction to present this part}

        \newpage

        \section{Map generation - a fast method}
        Present here what's done in \cite{RBG} : 
        - height map with a noise (might speak about libnoise) (topographical map)
        - then using a Whittaker diagram to compute the geographical map

        Present its advantages (easy to compute)
        and its drawbacks (cannot control the shape of the map easily)

        \newpage
        \section{Map generation - our method}

        Explain here our algorithm (voronoi, height tree that looks like a perlin noise)

        also present its advantages (geographical map -> topographical map, informations stored in the voronoi...)

        \newpage
        \section{Tesselation}
        Quick passage to explain how the tesselation used works.

	% % % % % % % % % % % % % % % % % % % % % % % % % % % % % % % % % % % %
	% % % % % % % % % % % % % % % % % % % % % % % % % % % % % % % % % % % %
	% % %
	% % %  Boids section
	% % %
	% % % % % % % % % % % % % % % % % % % % % % % % % % % % % % % % % % % %
	% % % % % % % % % % % % % % % % % % % % % % % % % % % % % % % % % % % %

	\newpage
	\skiptooddpage
	~~
	\vspace{1cm}

	\part{A better title that 'boids'}
        ~~
	
	\vspace{1cm}
	\section{A quick introduction to present this part}
	

        \newpage

	% % % % % % % % % % % % % % % % % % % % % % % % % % % % % % % % % % % %
	% % % % % % % % % % % % % % % % % % % % % % % % % % % % % % % % % % % %
	% % %
	% % %  End of the content
	% % %
	% % % % % % % % % % % % % % % % % % % % % % % % % % % % % % % % % % % %
	% % % % % % % % % % % % % % % % % % % % % % % % % % % % % % % % % % % %

	\appendix
	\newpage
	
	\addcontentsline{toc}{section}{References}
	
	%\bibliographystyle{apalike}
	\bibliographystyle{unsrt}
        \nocite{*}
	{\normalsize
          \bibliography{Report_Ecosystem}}
\end{spacing}
\end{document}
